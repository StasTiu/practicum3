\documentclass{article}
\usepackage[english,russian]{babel}
\usepackage[utf8]{inputenc}
\usepackage{amsmath}
\usepackage{amssymb}
\usepackage{amsthm}
\usepackage{graphicx}
\usepackage[margin=1in]{geometry}
\usepackage{fancyhdr}
\usepackage{blindtext}
\usepackage{url}
\usepackage{listings}
\usepackage{float}
\restylefloat{table}
\usepackage{diagbox}

\newtheorem{theorem}{Теорема}[section]
\newtheorem{corollary}{Следствие}[theorem]
\newtheorem{lemma}[theorem]{Лемма}
\newtheorem{pred}{Утверждение}
\theoremstyle{remark}
\newtheorem*{remark}{Замечание}
\newtheorem{definition}{Определение}
\theoremstyle{definition}
\newtheorem*{problem}{Задача}

\newcommand{\R}{\mathbb{R}}
\newcommand{\mc}[1]{\mathcal{#1}}
\newcommand{\tsf}[1]{\textsf{#1}}
\newcommand{\tbf}[1]{\textbf{#1}}

\title{Отчет по задаче практикума \\ <<Уравнения с частными производными>>}

\date{\today}
\author{Тиунова Анастасия, 407 группа}

\begin{document}
	\maketitle
	
	\section{Постановка задачи}
 	\begin{problem} \label{1}
Найти решение задачи (3.12)
$$
\begin{cases}
   \frac{\partial u}{\partial t} =\alpha \frac{\partial^2 u}{\partial x^2} - \left(1+x^2\right)u, \\
   \alpha \in \{1.0; 0.01\},\\
   x=0:\,\,\frac{\partial u}{\partial x}=0;\\
   x=1:\,\, u=1;\\
   t=0:\,\, u=1.
 \end{cases} $$
Воспользуемся методом четного продолжения относительно прямой $x=0$ (такое допустимо т.к. в нуле $\frac{\partial u}{\partial x}=0$ ), тогда задача сводится к:
$$
\begin{cases}
   \frac{\partial u}{\partial t} =\alpha \frac{\partial^2 u}{\partial x^2} - \left(1+x^2\right)u, \\
   \alpha \in \{1.0; 0.01\},\\
   x=-1:\,\,u=1;\\
   x=1:\,\, u=1;\\
   t=0:\,\, u=1.
 \end{cases} $$
\end{problem}

\section{Метод решения}

	Будем использовать метод сеток. Приближенное решение задачи ищем в виде сеточной функции, т.е. функции, определенной в каждом узле сетки $(N,M)$. Эта функция обозначается $\{u^n_m\}$. 
	
	Значение $u^n_m$ будем трактовать как приближенное значение функции $u(t,x)$ в узле $(t_n,x_m)$, т.е. 
	$$
	u^n_m \sim u(t_n,x_m).
	$$
	
	Сеточную функцию получим как решение разностного уравнения. Принятый способ разностной аппроксимации называют \emph{схемой}. Для решения нашей задачи будем использовать схему с весами, где вес  $\delta \in [0,1]$  будет выбран позднее:
	$$
	\boxed{ \frac{u_{m}^{n+1}-u^n_m}{\tau} = \alpha \left(\delta\frac{u^n_{m-1}-2u^n_m + u^n_{m+1}}{h^2}+(1-\delta)\frac{u^{n+1}_{m-1}-2u^{n+1}_m + u^{n+1}_{m+1}}{h^2}\right) + \frac{f^n_m+f^{n+1}_m}{2}}.
	$$
	
	Идея: последовательно выражать неизвестные сеточные функции из верхнего слоя через известные с нижних слоев с помощью разностного уравнения и граничных условий.
	\begin{itemize}
	\item[(I)]
	
	Рассмотрим уравнение из условия
	$$
	\frac{\partial u}{\partial t} =\alpha \frac{\partial^2 u}{\partial x^2} -\left(1+x^2\right)u.
	$$
	
	Запишем для него параметрическое семейство разностных схем с весом $\delta =\frac 1 2$:
	\begin{equation}
	\frac{u^{n+1}_{m}-u^n_m}{\tau} = \alpha\left(\frac 1 2 \frac{u^n_{m-1}-2u^n_m+u^n_{m+1}}{h^2}+(1-\frac 1 2)\frac{u^{n+1}_{m-1}-2u^{n+1}_m+u^{n+1}_{m+1}}{h^2}\right)-\left(1+(hm)^2\right)\frac{u^n_m+u^{n+1}_m}2
	\end{equation}
	
	
\item[(II)]	Изучим порядок аппроксимации в зависимости от веса $\delta$:\\
Разложим имеющиеся функции в ряд Тейлора в окрестности точки $(t_n,x_m)$:
	
	\begin{multline*}u^{n+1}_m= u(t_n+\tau,x_m)=u(t_n,x_m)+\tau u_t(t_n,x_m)+\frac 1 2 \tau^2 u_{tt}(t_n,x_m)+\frac 1 6 \tau^3 u_{ttt}(t_n,x_m)+o(\tau^3)
	\end{multline*}	
	\begin{multline*}u^n_{m\pm1}= u(t_n,x_m\pm h)=u(t_n,x_m)\pm hu_x(t_n,x_m)+\frac 1 2 h^2 u_{xx}(t_n,x_m)\pm\frac 1 6 h^3 u_{xxx}(t_n,x_m)+o(h^3)
	\end{multline*}
	\begin{multline*}u^{n+1}_{m\pm1}= u(t_n,x_m- h)=u(t_n,x_m)-hu_x(t_n,x_m)+\tau u_t(t_n,x_m)+\frac 1 2 h^2 u_{xx}(t_n,x_m)+\\
	+h\tau u_{xt}(t_n,x_m)+\frac 1 2 \tau^2 u_{tt}(t_n,x_m)+\frac 1 6 tau^3 u_{ttt}(t_n,x_m)\pm\frac 1 2 \tau^2hu_{ttx}(t_n,x_m)+\\
	+\frac 1 2 \tau h^2u_{txx}(t_n,x_m)\pm\frac 1 6h^3u_{xxx}(t_n,x_m)+o(h^3+\tau^3)
	\end{multline*}
	Подставим результаты в схему:
	$$
	u_t+o(\tau^2)=\alpha\left(\frac 1 2(u_{xx}+o(h^2))+(1-\frac 1 2)(u_{xx}+o(h^2))\right)-(1+x^2)u;
	$$
	$$
	u_t=\alpha u_{xx}-(1+x^2)u+o(h^2)+o(\tau^2);
	$$
\item[(III)]Изучим устойчивость схемы для $\delta=\frac 1 2$:\\
Варьированием (1) по $u^n_m$ получаем линейное уравнение на $\Delta u^n_m$:

\begin{multline*}
\frac{\Delta u^{n+1}_m-\Delta u^n_m}{\tau}=\alpha\left(\frac{1}{2}\frac{\Delta u^n_{m-1}-2\Delta u^n_m+\Delta u^n_{m+1}}{h^2}+(1-\frac{1}{2})\frac{\Delta u^{n+1}_{m-1}-2\Delta u^{n+1}_m+\Delta u^{n+1}_{m+1}}{h^2}\right)-\\
-(1+(hm)^2)\frac{\Delta u^n_m+ \Delta u^{n+1}_m}2.
\end{multline*}
	Замораживая коэффициенты в нем, приходим к линейному уравнению с постоянными коэффициентами, решение всегда имеет вид:
	\begin{equation}
	\Delta u^n_m=\lambda^n e^{ik\phi}u^0_0
	\end{equation}
	Подставим (3) в (2):
$$
\frac{\lambda-1}\tau=\alpha\left(\frac{1}{2}\frac{\cos(\phi)-1}{h^2}+(1-\frac{1}{2})\lambda\frac{\cos(\phi)-1}{h^2}\right)-(1+x^2);
$$	
$$
\lambda=\frac{1+\tau\alpha\frac{\cos(\phi)-1}{h^2}}{1-\tau\left(\alpha\frac{\cos(\phi)-1}{h^2}\right)}-\frac{\tau(1+x^2)}{1-\tau\left(\alpha\frac{\cos(\phi)-1}{h^2}\right)}\,\,\Rightarrow\,\,|\lambda|\leq 1+o(\tau)
$$	
$\Rightarrow$ схема безусловно устойчива.
	
	\item[(IV)]Из показанного выше, следует,что предложенная схема безусловно устойчива и имеет порядок аппроксимации  $o(\tau^2+h^2)$.
	$$
	\boxed{ \frac{u^{n+1}_{m}-u^n_m}{\tau} = \alpha\left(\frac 1 2 \frac{u^n_{m-1}-2u^n_m+u^n_{m+1}}{h^2}+(1-\frac 1 2)\frac{u^{n+1}_{m-1}-2u^{n+1}_m+u^{n+1}_{m+1}}{h^2}\right)-\left(1+(hm)^2\right)\frac{u^n_m+u^{n+1}_m}2}.
	$$
	Идея: составить систему линейных уравнений с трехдиагональной матрицей на неизвестные \\$u^{n+1}_{m-1}, u^{n+1}_m, u^{n+1}_{m+1}$ с коэффициентами и свободным членом, вычисленными с нижнего слоя с использованием граничных условий. \\
	Из граничных условий:
	\begin{itemize}
	\item[1)]$x=-1:\,\,u^n_m=1$
	\item[2)]$x=1:\,\,u^n_m=1$
	\item[3)]$t=0:\,\,u^n_m=1$
	\end{itemize}
	Тогда:
	\begin{multline*}
	m=1:\,\,\,u^{n+1}_m\left(\frac {1} {\tau}-\frac \alpha {2h^2}+\frac 1 2(1+(hm)^2)\right)
	-\frac \alpha {2h^2}u^{n+1}_{m+1}=\\
	=\frac{u^n_m}{\tau}+\frac{\alpha}{2}\frac{u^n_{m-1}-2u^n_m+u^n_{m+1}}{h^2}-(1+(hm)^2)\frac{u^n_m}{2}
	\end{multline*}	
	\begin{multline*}
	m\in\{2,..,M-3\}:\,\,\,-\frac \alpha {2h^2}u^{n+1}_{m-1}+u^{n+1}_m\left(\frac {1} {\tau}+\frac \alpha {h^2}+\frac 1 2(1+(hm)^2)\right)
	-\frac \alpha {2h^2}u^{n+1}_{m+1}=\\
	=\frac{u^n_m}{\tau}+\frac{\alpha}{2}\frac{u^n_{m-1}-2u^n_m+u^n_{m+1}}{h^2}-(1+(hm)^2)\frac{u^n_m}{2}
	\end{multline*}	
	\begin{multline*}
	m=M-2:\,\,\,-\frac \alpha {2h^2}u^{n+1}_{m-1}+u^{n+1}_m\left(\frac {1} {\tau}+\frac \alpha {h^2}+\frac 1 2(1+(hm)^2)\right)=\\
	=\frac \alpha {2h^2}+\frac{u^n_m}{\tau}+\frac{\alpha}{2}\frac{u^n_{m-1}-2u^n_m+u^n_{m+1}}{h^2}-(1+(hm)^2)\frac{u^n_m}{2}
	\end{multline*}		
	
	\end{itemize}
	
	\section{Вычислительный эксперимент}\label{Res}

    Ниже представлено вычесленное значение в точке максимума и минимума производной поtприразныхn,при $\alpha= 1$
	\begin{table}[H]
		\begin{center}
			\begin{tabular}{|c||c|c|c|c|c|}
			\hline
\diagbox[width=3em, height=2em]{$t$}{$x$} & $1$ & $\frac 4 5$ & $\frac 3 5$ & $\frac 2 5$ & $\frac 1 5$ \\
				\hline
					\hline
0.1 & 0.89823584 & 0.89406673  & 0.89654779 & 0.89786238 & 0.89598834\\
$t_{Max D_t} 0$ & 0.99981994 & 0.99984659  & 0.99986214 & 0.99986658 & 0.99985992\\
0.9 & 0.73141937 & 0.66655632  & 0.63197447 & 0.62251293 & 0.63677409\\
\hline
			\end{tabular}
		\end{center}
				\caption{$\tau=\frac {h^2} 3, m=50$}
	\end{table}
	
	\begin{table}[H]
		\begin{center}
			\begin{tabular}{|c||c|c|c|c|c|}
			\hline
\diagbox[width=3em, height=2em]{$t$}{$x$} & $1$ & $\frac 4 5$ & $\frac 3 5$ & $\frac 2 5$ & $\frac 1 5$ \\
				\hline
					\hline

0.1 & 0.89881706 & 0.89418373  & 0.89649535 & 0.89790958 & 0.89622441\\
$t_{Max D_t} 0$ & 0.99995483 & 0.99996149  & 0.99996544 & 0.99996666 & 0.99996517\\
0.9 & 0.73379455 & 0.66863746  & 0.63340792 & 0.62295348 & 0.63576145\\
\hline
			\end{tabular}
		\end{center}
				\caption{$\tau=\frac {h^2} 3, m=100$}
	\end{table}
	\begin{table}[H]
		\begin{center}
			\begin{tabular}{|c||c|c|c|c|c|}
			\hline
\diagbox[width=3em, height=2em]{$t$}{$x$}& $1$ & $\frac 4 5$ & $\frac 3 5$ & $\frac 2 5$ & $\frac 1 5$ \\
				\hline
					\hline
0.1 & 0.89896643 & 0.89420539  & 0.89647078 & 0.89791855 & 0.89629163\\
$t_{Max D_t} 0$ & 0.99997990 & 0.99998286  & 0.99998462 & 0.99998518 & 0.99998454\\
0.9 & 0.73448739 & 0.66924910  & 0.63381622 & 0.62304101 & 0.63537489\\
\hline
			\end{tabular}
		\end{center}
				\caption{$\tau=\frac {h^2} 3, m=150$}
\end{table}
	
	Изучим главный член аппроксимации в опорных точках, ниже представлены значения $\Delta u $в них:
	\begin{table}[H]
		\begin{center}
			\begin{tabular}{|c||c|c|c|c|c|}
			\hline
\diagbox[width=3em, height=2em]{$t$}{$x$} & $1$ & $\frac 4 5$ & $\frac 3 5$ & $\frac 2 5$ & $\frac 1 5$ \\
				\hline
					\hline
0.1 & -0.00058123 & -0.00011700  & 0.00005245 & -0.00004720 & -0.00023607\\
$t_{Max D_t}$ & -0.00013489 & -0.00011490  & -0.00010330 & -0.00010008 & -0.00010524\\
0.9 & -0.00237518 & -0.00208114  & -0.00143344 & -0.00044055 & 0.00101264\\

\hline
			\end{tabular}
		\end{center}
				\caption{$\tau=\frac {h^2} 3,\Delta u$ при $m=50 -> m=100$}
	\end{table}
	\begin{table}[H]
		\begin{center}
			\begin{tabular}{|c||c|c|c|c|c|}
			\hline
\diagbox[width=3em, height=2em]{$t$}{$x$} & $1$ & $\frac 4 5$ & $\frac 3 5$ & $\frac 2 5$ & $\frac 1 5$ \\
				\hline
					\hline
0.1 & -0.00014937 & -0.00002166  & 0.00002457 & -0.00000897 & -0.00006722\\
$t_{Max D_t}$ & -0.00002507 & -0.00002137  & -0.00001919 & -0.00001852 & -0.00001938\\
0.9 & -0.00069283 & -0.00061164  & -0.00040831 & -0.00008752 & 0.00038656\\
\hline
			\end{tabular}
		\end{center}
				\caption{$\tau=\frac {h^2} 3,\Delta u$ при $m=100 -> m=150$}
	\end{table}
	
	Итого, для $\frac{u(m=50)-u(m=100)}{u(m=100)-u(m=150)}$ в опорных точках:
	\begin{table}[H]
		\begin{center}
			\begin{tabular}{|c||c|c|c|c|c|}
			\hline
\diagbox[width=3em, height=2em]{$t$}{$x$}  & $1$ & $\frac 4 5$ & $\frac 3 5$ & $\frac 2 5$ & $\frac 1 5$ \\
				\hline
					\hline
0.1 & 3.89112728 & 5.40256135  & 2.13508599 & 5.26259202 & 3.51178444\\
$t_{Max D_t}$ & 5.37980063 & 5.37667767  & 5.38377453 & 5.40297738 & 5.43101757\\
0.9 & 3.42821751 & 3.40254641  & 3.51069205 & 5.03363792 & 2.61960198\\ 
\hline
			\end{tabular}
		\end{center}
	\end{table}
	
	Аналогично для $\alpha=0.01$:
	\begin{table}[H]
		\begin{center}
			\begin{tabular}{|c||c|c|c|c|c|}
			\hline
\diagbox[width=3em, height=2em]{$t$}{$x$} & $1$ & $\frac 4 5$ & $\frac 3 5$ & $\frac 2 5$ & $\frac 1 5$ \\
				\hline
					\hline
0.1 & 0.87359323 & 0.89123356  & 0.90168780 & 0.90469718 & 0.90018686\\
$t_{Max D_t} 0$ & 0.99981996 & 0.99984661  & 0.99986216 & 0.99986660 & 0.99985994\\
0.9 & 0.29535552 & 0.35272414  & 0.39132466 & 0.40311003 & 0.38556182\\
\hline
			\end{tabular}
		\end{center}
				\caption{$\tau=\frac {h^2} 3, m=50$}
	\end{table}
	
	\begin{table}[H]
		\begin{center}
			\begin{tabular}{|c||c|c|c|c|c|}
			\hline
\diagbox[width=3em, height=2em]{$t$}{$x$} & $1$ & $\frac 4 5$ & $\frac 3 5$ & $\frac 2 5$ & $\frac 1 5$ \\
				\hline
					\hline
0.1 & 0.87317738 & 0.89081313  & 0.90141794 & 0.90473469 & 0.90068254\\
$t_{Max D_t} 0$ & 0.99995483 & 0.99996149  & 0.99996544 & 0.99996666 & 0.99996517\\
0.9 & 0.29413794 & 0.35124527  & 0.39028283 & 0.40325867 & 0.38745649\\
\hline
			\end{tabular}
		\end{center}
				\caption{$\tau=\frac {h^2} 3, m=100$}
	\end{table}
	\begin{table}[H]
		\begin{center}
			\begin{tabular}{|c||c|c|c|c|c|}
			\hline
\diagbox[width=3em, height=2em]{$t$}{$x$}& $1$ & $\frac 4 5$ & $\frac 3 5$ & $\frac 2 5$ & $\frac 1 5$ \\
				\hline
					\hline
0.1 & 0.87303833 & 0.89067196  & 0.90132521 & 0.90474153 & 0.90083822\\
$t_{Max D_t} 0$ & 0.99997990 & 0.99998286  & 0.99998462 & 0.99998518 & 0.99998454\\
0.9 & 0.29372975 & 0.35074990  & 0.38992539 & 0.40328576 & 0.38805324\\
\hline
			\end{tabular}
		\end{center}
				\caption{$\tau=\frac {h^2} 3, m=150$}
\end{table}
	
	Изучим главный член аппроксимации в опорных точках, ниже представлены значения $\Delta u $в них:
	\begin{table}[H]
		\begin{center}
			\begin{tabular}{|c||c|c|c|c|c|}
			\hline
\diagbox[width=3em, height=2em]{$t$}{$x$} & $1$ & $\frac 4 5$ & $\frac 3 5$ & $\frac 2 5$ & $\frac 1 5$ \\
				\hline
					\hline
0.1 & 0.00041586 & 0.00042043  & 0.00026985 & -0.00003751 & -0.00049568\\
$t_{Max D_t}$ & -0.00013487 & -0.00011488  & -0.00010328 & -0.00010006 & -0.00010523\\
0.9 & 0.00121758 & 0.00147887  & 0.00104183 & -0.00014865 & -0.00189467\\
\hline
			\end{tabular}
		\end{center}
				\caption{$\tau=\frac {h^2} 3,\Delta u$ при $m=50 -> m=100$}
	\end{table}
	\begin{table}[H]
		\begin{center}
			\begin{tabular}{|c||c|c|c|c|c|}
			\hline
\diagbox[width=3em, height=2em]{$t$}{$x$} & $1$ & $\frac 4 5$ & $\frac 3 5$ & $\frac 2 5$ & $\frac 1 5$ \\
				\hline
					\hline
0.1 & 0.00013905 & 0.00014117  & 0.00009274 & -0.00000683 & -0.00015567\\
$t_{Max D_t}$ & -0.00002507 & -0.00002137  & -0.00001919 & -0.00001852 & -0.00001938\\
0.9 & 0.00040819 & 0.00049537  & 0.00035744 & -0.00002709 & -0.00059675\\
\hline
			\end{tabular}
		\end{center}
				\caption{$\tau=\frac {h^2} 3,\Delta u$ при $m=100 -> m=150$}
	\end{table}
	
	Итого, для $\frac{u(m=50)-u(m=100)}{u(m=100)-u(m=150)}$ в опорных точках:
	\begin{table}[H]
		\begin{center}
			\begin{tabular}{|c||c|c|c|c|c|}
			\hline
\diagbox[width=3em, height=2em]{$t$}{$x$}  & $1$ & $\frac 4 5$ & $\frac 3 5$ & $\frac 2 5$ & $\frac 1 5$ \\
				\hline
					\hline
0.1 & 2.99071876 & 2.97815568  & 2.90984076 & 5.48961893 & 3.18411450\\
$t_{Max D_t}$ & 5.37933200 & 5.37612763  & 5.38316217 & 5.40234397 & 5.43041343\\
0.9 & 2.98285475 & 2.98536772  & 2.91470846 & 5.48778344 & 3.17494935\\
\hline
			\end{tabular}
		\end{center}
	\end{table}
	
Нетрудно заметить, что вычислительный эксперимент соответствует теоретическим результатам, а именно:
\begin{itemize}
\item[1)] увеличение числа шагов приводит к более точному, (ограниченному) результату, что соответствует устойчивости схемы и граничных условий.
\item[2)] изменение функции, при увеличении числа шагов, уменьшается пропорционально квадрату длине малых отрезков, на которые разбивается изначальный, что соотносится с порядком аппроксимации, полученным теоретически.
\end{itemize}
	\section{Листинг программы}
	\begin{lstlisting}[language=C]
#include <math.h>
#include <stdio.h>
#include <stdlib.h>
void RESHI(double *a, double *x, double *b,int N);
int Max(double *p,int nn, int M);
int Min(double *p,int w,int nn, int M);
void makeAandB(double *p,double *A, double *B, double t, double h,double a,int j,int M);


double diff (double a, double b, double p[], int nn, int M, int T,double strMax[],int numMax,double str01[],double str09[],double s)
{
    double t, n1, M1, h;
    n1=nn;
    M1=M;
    t=T/(n1-1);
    h=2/(M1-1);
    for(int i=M; i<=M*2-1; i++){p[i]=1;}
    for(int j=nn-2; j>=0; j--)
    {
        double *A,*B,*X;
        B = (double *) malloc ((M - 2) * sizeof (double));
        X = (double *) malloc ((M - 2) * sizeof (double));
        A = (double *) malloc ((M - 2) * 3 * sizeof (double));

        makeAandB(p,A,B,t,h,a,j,M);
        RESHI(A,X,B,M-2);
        for(int i=1;i<=M-2;i++)
        {
            p[i]=X[i-1];
        }
        p[0]=1;
        p[M-1] = 1;
        if(j==(nn-1)/10){
            for(int i=0;i<5;i++)
                str09[i]=p[(i*M)/10+(M/5) ];
            // printf("%d////", j);
            // printf("\n");
        }
        if(j==9*(nn-1)/10){
            for(int i=0;i<5;i++)
                str01[i]=p[(i*M)/10+(M/5)];
            //printf("%d////", j);
        }
        for( int i=0;i<M-1;i++)

        {
            if(fabs(p[i]-p[M+i])>=fabs(s)){s=p[i]-p[M+i];numMax=j;
                //printf("%d////", numMax);
                //printf("%.8f   ", p[i]-p[M+i]);
                for(int k=0; k<5;k++){
                    strMax[k]=p[(k*M)/10+(M/5)];
                }
            }
        }
        for (int i = 0; i <= M - 1; i++){

            //printf("%.8f   ", p[i+M]);
            p[i+M]=p[i];
        }
        //printf("\n");


    }


    /*for(int i=0; i<=2*M-1; i++)
    {
        if (i%M==0) printf("\n");
        printf("%.8f   ", p[i]);
    }
    printf("\n");
    printf("\n");/*
       for(int i=0; i<=nn*M-1; i++)
       {
       if (i%M==0) printf("\n");
       printf("%.8f   ", Dx[i]);
       }
       printf("\n");*/
    return 0;
}


int main ()
{
    int nn=10,M=10,T=1,numMax,M1,nn1, numMax1, nn2, numMax2, M2;
    double  s=0,b = 0.1, a = 0.01,strMax[5],str01[5],str09[5],strMax1[5],str011[5],str091[5],n2,t2,str012[5],str092[5],n3,t3,strMax2[5];
    for(int j=0; j<1;j++){
        M = 50*(j+1);
        nn=M*M*3;
        double *p,*p1,*p2;
        p = (double *)malloc((2*M)* sizeof (double));
        diff(a,b,p,nn,M,T,strMax,numMax,str01,str09,s);
        /*printf ( "0.1 & %.8f & %.8f  & %.8f & %.8f & %.8f",str01[0],str01[1],str01[2],str01[3],str01[4]);
        printf("\\\\");
        printf("\n");
        printf ( "$t_{Max D_t} %d$ & %.8f & %.8f  & %.8f & %.8f & %.8f",numMax, strMax[0],strMax[1],strMax[2],strMax[3],strMax[4]) ;
        printf("\\\\");
        printf("\n");
        printf ( "0.9 & %.8f & %.8f  & %.8f & %.8f & %.8f",str09[0],str09[1],str09[2],str09[3],str09[4]);
        printf("\\\\");
        printf("\n");*/

        M1 = 100*(j+1);
        nn1=M1*M1*3;
        p1 = (double *)malloc((2*M1)* sizeof (double));
        diff(a,b,p1,nn1,M1,T,strMax1,numMax1,str011,str091,s);


        M2 = 150*(j+1);
        nn2=M2*M2*3;
        p2 = (double *)malloc((2*M2)* sizeof (double));
        diff(a,b,p2,nn2,M2,T,strMax2,numMax2,str012,str092,s);

        printf ( "0.1 & %.8f & %.8f  & %.8f & %.8f & %.8f",str01[0]-str011[0],str01[1]-str011[1],
                 str01[2]-str011[2],str01[3]-str011[3],str01[4]-str011[4]);
        printf("\\\\");
        printf("\n");
        printf ( "$t_{Max D_t}$ & %.8f & %.8f  & %.8f & %.8f & %.8f", strMax[0]-strMax1[0],strMax[1]-strMax1[1],
                 strMax[2]-strMax1[2],strMax[3]-strMax1[3],strMax[4]-strMax1[4]) ;
        printf("\\\\");
        printf("\n");
        printf ( "0.9 & %.8f & %.8f  & %.8f & %.8f & %.8f",str09[0]-str091[0],str09[1]-str091[1],
                 str09[2]-str091[2],str09[3]-str091[3],str09[4]-str091[4]);
        printf("\\\\");

        printf("\n");
        printf("\n");
        printf("\n");

        printf ( "0.1 & %.8f & %.8f  & %.8f & %.8f & %.8f",str011[0]-str012[0],str011[1]-str012[1],str011[2]-str012[2],
                 str011[3]-str012[3],str011[4]-str012[4]);
        printf("\\\\");
        printf("\n");
        printf ( "$t_{Max D_t}$ & %.8f & %.8f  & %.8f & %.8f & %.8f", strMax1[0]-strMax2[0],strMax1[1]-strMax2[1],
                 strMax1[2]-strMax2[2],strMax1[3]-strMax2[3],strMax1[4]-strMax2[4]) ;
        printf("\\\\");
        printf("\n");
        printf ( "0.9 & %.8f & %.8f  & %.8f & %.8f & %.8f",str091[0]-str092[0],str091[1]-str092[1],str091[2]-str092[2],
                 str091[3]-str092[3],str091[4]-str092[4]);
        printf("\\\\");

        printf("\n");
        printf("\n");
        printf("\n");

        printf ( "0.1 & %.8f & %.8f  & %.8f & %.8f & %.8f",(str01[0]-str011[0])/(str011[0]-str012[0]),
                 (str01[1]-str011[1])/(str011[1]-str012[1]),(str01[2]-str011[2])/(str011[2]-str012[2]),
                 (str01[3]-str011[3])/(str011[3]-str012[3]),(str01[4]-str011[4])/(str011[4]-str012[4]));
        printf("\\\\");
        printf("\n");
        printf ( "$t_{Max D_t}$ & %.8f & %.8f  & %.8f & %.8f & %.8f", (strMax[0]-strMax1[0])/( strMax1[0]-strMax2[0]),
                 (strMax[1]-strMax1[1])/(strMax1[1]-strMax2[1]),(strMax[2]-strMax1[2])/(strMax1[2]-strMax2[2]),
                 (strMax[3]-strMax1[3])/(strMax1[3]-strMax2[3]),(strMax[4]-strMax1[4])/(strMax1[4]-strMax2[4]));
        printf("\\\\");
        printf("\n");
        printf ( "0.9 & %.8f & %.8f  & %.8f & %.8f & %.8f",(str09[0]-str091[0])/(str091[0]-str092[0]),
                 (str09[1]-str091[1])/(str091[1]-str092[1]),(str09[2]-str091[2])/(str091[2]-str092[2]),
                 (str09[3]-str091[3])/(str091[3]-str092[3]),(str09[4]-str091[4])/(str091[4]-str092[4]));
        printf("\\\\");


    }

    return 0;
}



int Max(double *p,int nn, int M)
{
    double s=0.0;
    int i,k;
    for(i=0;i<nn*M;i++)
    {
        if(p[i]>=s && !(p[i]>=100 && p[i]<=100)){s=p[i];k=i;}
    }
    return k;
}
int Min(double *p,int w,int nn, int M)
{
    double s=2.0;
    int i,k;
    for(i=0;i<nn*M;i++)
    {
        if(p[i]<s && i>=w*M){s=p[i];k=i;}
    }
    return k;
}
void RESHI(double *a, double *x, double *b,int N)
{
    double *s,*p,*y;
    s = (double *) malloc (N * sizeof (double));
    p = (double *) malloc (N* sizeof (double));
    y = (double *) malloc (N* sizeof (double));
    y[0]=a[0];
    s[0]=-a[1]/y[0];
    p[0]=b[0]/y[0];
    for(int i=1;i<N-1;i++)
    {
        y[i]=a[3*i]+a[3*i+2]*s[i-1];
        s[i]=-a[3*i+1]/y[i];
        p[i]=(b[i]-a[3*i+2]*p[i-1])/y[i];
    }
    y[N-1]=a[(N-1)*3]+a[(N-1)*3+2]*s[N-2];
    p[N-1]=(b[N-1]-a[(N-1)*3+2]*p[N-2])/y[N-1];
    x[N-1]=p[N-1];
    for(int i=N-2;i>=0;i--)
    {
        x[i]=s[i]*x[i+1]+p[i];
    }
    return;
}
void makeAandB(double *p,double *A, double *B, double t, double h,double a, int j, int M)
{
    A[0]=1/t+a/(h*h)+(1-h)*(1-h);
    A[1]=-a/(2*h*h);
    A[2] = 0;
    A[(M-2)*3-2] = 0;
    A[(M-2)*3-1]=-a/(2*h*h);
    A[(M-2)*3-3]=1/t+a/(h*h)+(1-h)*(1-h);

    B[0]=a/(2*h*h)+p[(1)*(M)+1]*(1/t-(1+(1-h)*(1-h)))+a*(p[(1)*M]-2*p[(1)*M+1]+p[(1)*M+2])/(2*h*h);
    B[M-3]= a/(2*h*h)+a*(p[(1)*M+M-3]-2*p[(1)*M+M-2]+p[(1)*M+M-1])/(2*h*h)+p[(1)*(M)+M-2]*(1/t-(1+(1-h)*(1-h)));

    for(int i=1; i<M-3; i++)
    {
        A[3*i+2]=-a/(2*h*h);
        A[3*i]=1/t+a/(h*h)+(1+(h*(i+1)-1)*(h*(i+1)-1))/2;
        A[3*i+1]=-a/(2*h*h);
        B[i]=p[(1)*(M)+1+i]*(1/t-(1+(h*(i+1)-1)*(h*(i+1)-1))/2)+a*(p[(1)*M+i]-2*p[(1)*M+1+i]+p[(1)*M+2+i])/(2*h*h);
    }
    return;
}

	\end{lstlisting}
\end{document}